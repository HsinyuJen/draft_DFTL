
%%%%%%%%%%%%%%%%%%%%%%% file typeinst.tex %%%%%%%%%%%%%%%%%%%%%%%%%
%
% This is the LaTeX source for the instructions to authors using
% the LaTeX document class 'llncs.cls' for contributions to
% the Lecture Notes in Computer Sciences series.
% http://www.springer.com/lncs       Springer Heidelberg 2006/05/04
%
% It may be used as a template for your own input - copy it
% to a new file with a new name and use it as the basis
% for your article.
%
% NB: the document class 'llncs' has its own and detailed documentation, see
% ftp://ftp.springer.de/data/pubftp/pub/tex/latex/llncs/latex2e/llncsdoc.pdf
%
%%%%%%%%%%%%%%%%%%%%%%%%%%%%%%%%%%%%%%%%%%%%%%%%%%%%%%%%%%%%%%%%%%%


\documentclass[runningheads,a4paper]{llncs}

\usepackage{amssymb}
\setcounter{tocdepth}{3}
\usepackage{graphicx}
\usepackage{verbatim}%长篇注释宏包
\usepackage{mathrsfs}
\usepackage{mathtools}
\usepackage{multirow}
\usepackage{cite}
\usepackage{amsmath}
\usepackage{url}
\urldef{\mailsa}\path|{alfred.hofmann, ursula.barth, ingrid.haas, frank.holzwarth,|
\urldef{\mailsb}\path|anna.kramer, leonie.kunz, christine.reiss, nicole.sator,|
\urldef{\mailsc}\path|erika.siebert-cole, peter.strasser, lncs}@springer.com|    
\newcommand{\keywords}[1]{\par\addvspace\baselineskip
\noindent\keywordname\enspace\ignorespaces#1}


\newcommand{\reffig}[1]{Fig. \ref{#1}}
\newcommand{\refsec}[1]{Section \ref{#1}}
% \newcommand{\refeq}[1]{Eq. \ref{#1}}
\newcommand{\reftab}[1]{Table \ref{#1}}




\begin{document}

\mainmatter  % start of an individual contribution

% first the title is needed
\title{Background Subtraction based on Superpixels under Multi-Scale in
Complex Scenes}

% a short form should be given in case it is too long for the running head
\titlerunning{Lecture Notes in Computer Science: Authors' Instructions}

% the name(s) of the author(s) follow(s) next
%
% NB: Chinese authors should write their first names(s) in front of
% their surnames. This ensures that the names appear correctly in
% the running heads and the author index.
%
\author{Alfred Hofmann
\thanks{Please note that the LNCS Editorial assumes that all authors have used
the western naming convention, with given names preceding surnames. This determines
the structure of the names in the running heads and the author index.}%
\and Ursula Barth\and Ingrid Haas\and Frank Holzwarth\and\\
Anna Kramer\and Leonie Kunz\and Christine Rei\ss\and\\
Nicole Sator\and Erika Siebert-Cole\and Peter Stra\ss er}
%
\authorrunning{Lecture Notes in Computer Science: Authors' Instructions}
% (feature abused for this document to repeat the title also on left hand pages)

% the affiliations are given next; don't give your e-mail address
% unless you accept that it will be published
\institute{Springer-Verlag, Computer Science Editorial,\\
Tiergartenstr. 17, 69121 Heidelberg, Germany\\
\mailsa\\
\mailsb\\
\mailsc\\
\url{http://www.springer.com/lncs}}

%
% NB: a more complex sample for affiliations and the mapping to the
% corresponding authors can be found in the file "llncs.dem"
% (search for the string "\mainmatter" where a contribution starts).
% "llncs.dem" accompanies the document class "llncs.cls".
%

\toctitle{Lecture Notes in Computer Science}
\tocauthor{Authors' Instructions}
\maketitle


\begin{abstract}
Background subtraction in complex scenes has been a challenging problem
in computer vision.
% Previous work
Most previous algorithms analyzed the variation of pixels or region for
background subtraction, which ignored the neighbor information and
similarity between pixels themselves.
% our work
In this paper, a novel background subtraction method based on
superpixels under multi-scale (SPMS) in complex scenes is proposed.
% 说明方法新颖的地方,优点
Since superpixel implied the neighbor information between similar
pixels, proposed approach achieves encouraging robustness in complex
scenes including adverse weather and dynamic scenes.
% 具体的方法
In our SPMS, pixels are clustered into several superpixels according to
the similarity between them.
% 每一个超像素的标记方法
Each super-pixel is subtracting from the respective zone of background
image to label as foreground or background.
% background image 如何得到
In particular, the background image is learned from the statistic of
pixels' intensity during video sequence.
% 多尺度
Moreover, in order to improve the robustness and efficiency of proposed
approach, the integration of foregrounds captured in different scales is
used as the final result of SPMS.
% 实验结果
The experiments on standard benchmark demonstrate the encouraging
performance of proposed approach in comparison with some
state-of-thee-art approaches.
\keywords{Background Subtraction, Motion, Superpixels, Multi-Scale}
\end{abstract}


\end{document}

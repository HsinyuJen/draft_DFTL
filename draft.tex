\documentclass[journal]{IEEEtran}

\usepackage{cite}
\usepackage{amsmath}
\usepackage{verbatim}
\usepackage{multirow}
\usepackage[unicode,pdftex]{hyperref}

\usepackage{xcolor}


\ifCLASSINFOpdf
\usepackage[pdftex]{graphicx}
\else
\fi

\hyphenation{op-tical net-works semi-conduc-tor}

\newcommand{\reffig}[1]{Fig. \ref{#1}}
\newcommand{\refsec}[1]{Section \ref{#1}}
\newcommand{\refeq}[1]{Eq. \ref{#1}}
\newcommand{\reftab}[1]{Table \ref{#1}}

\newcommand{\tabincell}[2]{\begin{tabular}{@{}#1@{}}#2\end{tabular}}
\newcommand{\stodo}[1]{\todo[size=\tiny]{#1}}


\DeclareMathOperator*{\argmax}{argmax}
\DeclareMathOperator*{\argmin}{argmin}


\definecolor{level2}{RGB}{255,255,255}
\definecolor{level3}{RGB}{10,10,10}
\definecolor{revised}{RGB}{100,101,140}
\definecolor{continue}{RGB}{255,0,0}
\definecolor{filltext}{RGB}{0,255,255}

\begin{document}
\title{Background Subtraction via Deep Variation Transformation}

\author{Yongxin Ge, 
        Xinyu Ren, 
        Chenqiu Zhao}

% XXX there is no such command of authorruning!!!!!!
% XXX 根本就没有authorruning 命令!!!
% \authorrunning{Lecture Notes in Computer Science: Chongqing University}


\maketitle



\begin{abstract}
%什么是背景检测 避免多次使用同一个词 generally
Background subtraction is fundamentally considered as the binary classification of pixels in a video stream. 
%之前方法 下一句用的focus on加ing,这就用ed,形成对比
Previous works generally proposed the artificial model with the utilization of lowlevel or hand-crafted features to analyze the variation of pixels' observation in time sequence.
% 为什么之前方法会失败
However, due to the complexity and diversity of nature,
the variation of observations becomes so hard to classify by hand-crafted features.
    %in complexly natural scenes.
%我们的改进
In this paper, we focus on transforming the obserations sequence into another easier classfied space,
and a novel background subtraction method based on Deep Variation Transformation (DVT) is proposed.
% XXX 为什么是networks,不是network呢?
In the DVT model,
the fully convolutional network (FCN) is untlized to transform the sequence of pixels' observations into a new representation of pixels' observations with more obvious features for classfication.
%
In particlar, 
a the variation of observation represented by pixel sequence is reshaped into a image patch as the input of FCN network.
%
Then, the output of network is classfied and reverted to the corresponding sequence for the binary classfication of oberservations' variation.
%
Benefied from the ingenious utilization of FCN network leading by our clear congnition of essence about background subtractoin problem,
proposed approached adaptively generate superior peroformances in diversly complex scenes.
%
Comprehensive experiments in standard benchmarks demonstrate the superiority of proposed approach compared with state-of-the-art methods including 
% both deep learning and traditional methods.
% 
% 
% 
% videos are divided into fixed length image stacks by temporal sampling. 
%     Next, image stacks is broken down into pixel observation matrixes as the input of the FCN to learn the patterns of pixel variation. 
%     Each of the pixel matrixes contains a piece of temporal information about pixel variation over a period of time.  
% The architecture of our FCN is devised from the semantic segmentation problem. 
% %效果
% Experiment shows that our model has a strong learning ability to the patterns of those permutations of pixels’ observations. 
%     Compared with some state-of-the-art methods, both deep learning and traditional methods, the proposed approach outperform its conventional counterparts significantly owing to the utilizing of temporal information.
\end{abstract}

\begin{IEEEkeywords} 
    Background Subtraction, Feature Transformation, Deep Learning,
\end{IEEEkeywords}

\IEEEpeerreviewmaketitle

\section{Introduction}
%背景检测的背景介绍
Background subtraction or foreground object detection, as a fundamental problem in computer vision, has been much discussed with the increasing number of outdoor cameras over the last few decades.
It is widely used as the pre-processing step of video processing, which can help us efficiently mark the region of interest, e.g. vehicles and humans, thus saving us huge amount of computing resources.
Typically, Background subtraction can be viewed as a binary classification that assigns each pixel in a video sequence with a label, for either belonging to the background or foreground scene. 

%实际应用中的挑战
Existed methods have already achieved well performances in the scenes of low diversity or complexity, such as the indoor scenes. However, background subtraction is still unsolved because of the complexity and the diversity of natural scenes.
There are many forms of natrual scenes, for example, camera jitter, dynamic background, bad weather, illumination changing, intermittent object motion. In these situations, the backgrounds are no longer being static, while in fact, the background can be dynamic and complex, which brings some severe challenges to the traditional methods.
\begin{figure}[!t]	% FIGURE: figure/fig1 
\centering
    \includegraphics[width=\linewidth]{figure/idea.pdf}
    % 图例里描述一下。
    \caption{XXX.}
    \label{idea}
\end{figure}


%本文的想法和优势
The previous methods have already made great progress in generating some sophisticated background models with given videos. Unfortunately, some instructive clues have been long put aside and neglected, since there is no efficient way to deal with them. For instance, the statistical methods model the pixel-wise distribution of historical observations over time, while having no concerns over the sequential information in a video stream. To solve this problem, we bring up a novel conception of pixel variation transformation, wherein the historical observations of each pixel are conceived as a unity, the pixel variation. Each piece of pixel variation contains the complete temproal information of historical observations, which turns out to be our advantage that not only the distribution of observations, but also the temporal coherence plays an important role in the task.

%具体介绍
In this paper, we propose a novel Deep Variation Transformation Learning (DVTL) model for background subtraction in diversely natural scenes. In the DPVTL model, a Fully Convolutinal Network(FCN) is applied to learn the patterns of the pixel variation and find a transformation which guarantee the linear separability of transformed pixel variation generated by mapping in a new space. Our contributions can be summarized as follows:
\begin{itemize}
\item We proposed the pixel observation matrix to describe the pixel variation, which encode both the intensity distribution and sequential information over time. The observation matrix is obtained by reshaping the vector of a pixel's historical observations, subtracted from the image stacks. Since the pixel sequences are extracted from individual pixels, a large number of pixel matrixes may be extracted from only a single image stack, which promise that sufficient training data can be obtained with limited groundtruth.
\item We propose a deep neural network for variation transformation. Our FCN network is trained to learn the pixel variation and generate a new representation in a new feature space. We take advantage of the strong learning ability of FCN to learn an end-to-end representation of the pixel observation matrixes in a new space where the pixels can be easily classified to background and foreground.
\end{itemize}

%文章结构
The outline of this paper is as follows: In Section II, we give a brief discussion about the early and recent relevant works. The details of the variation transformation are presented in Section III. The proposed fully convolutional network is illustrated in Section IV, followed by experiments and result comparison in Section V and conclude the paper in Section VI. 
\section{Related Work}
% XXX 引用文章的方式,Bouwmans2014 在ref.bib文件里,不懂去问曲师兄!!!!
\cite{Bouwmans2014}
Over the last few decades, background subtraction has been well studied. Meanwhile, a huge number of background modeling methods have been proposed. 
These methods can be broadly categorized into pixels-based, region-based and Learning-based methods.
\subsection{pixels-based methods}
Pixel-based techniques assume that the historical observations over time are independent at each pixel. Based on low-level features, such as color and gradients, these methods are computationally efficient and easy to deploy. However, they demonstrate a comparatively poor performance in modelling some challenging scenes, like illumination changing and intermittent object motion.

%GMM
The most famous pixels-based methods are Gaussian Mixture Model (GMM) methods, which utilize a mixture of weighted Gaussians to model the probability distribution of each pixel over time. Pixels are considered to be background if there exists a Gaussian includes their values with sufficient evidence. 
%GMM+
Zoran Zivkovic extend the GMM through the use of recursive equations, where the parameters keep constantly updating and the number of components are adaptable to each pixel. 
%codebook
Another popular algorithms in pixel based category are based on Codebook. In [], Kim et al present the codebook method to record the sampling background values at each pixel, which can be seemed as a compressed representation of background model. The final foreground is detected by a distance measurement in a cylindrical color model.
%KDE
A non-parametric background model is proposed in []. Elgammal et al. assume that each background pixel is drawn from a Probability Distribution Function(PDF). The PDF for each pixel is estimated with Kernel Density Estimation(KDE). 
%Vibe
Another non-parametric method is proposed by O. Barnich, called the Visual Background Extractor (ViBe). ViBe is a sample based method, which consists of pixel samples from the video stream. Each pixel in the current frame is compared with N pixel samples from the corresponding background model. A pixel is labeled as the foreground only when there exist at least K samples with a distance to itself within a certain range R. 
%PBAS
To adaptively update the parameters, Hofmann et al. [] improve the ViBe by presenting an adaptive threshold R(x), which depended on the pixel position and a background dynamics metric.
%
\subsection{region-based approach}
Region-based approaches assume that the neighboring pixels have a similar variation as the pixel itself. Hence the spatial correlation is taken into consideration to refine the pixel-level classification.

%(Regularized region-based)
A region-based MoG is proposed by Sriram Varadarajan et al. Their model are derived from expectation maximization theory, which takes into consideration neighboring pixels while generating the model of the observed scene.
% Spatiotemporal GMM
Another GMM based method named Spatiotemporal GMM algorithm was proposed in []. Mingliang Chen et al. combine the GMM with constrains of temporal and spatial information from the optical flow and superpixels.
%(Bayesian Modeling)
In [], Yaser Sheikh et al. introduce a MAP-MRF framework, which incorporate the pixel location into background and foreground KDEs for the detection based on spatial context.
%(Robust Foreground Object ICPR2010 Vikas Reddy )
In [], given frames are divided into overlapping blocks. Each block is sequentially processed by an adaptive multi-stage classifier, which consists of a likelihood evaluation, an illumination invariant measure, and a temporal correlation check.
%( Robust Region-Based ICPR)
Mohammad Izadi et al. present a robust region-based approach, which generates a pair of foreground maps based on gradient and color. Any foreground region that does not exist in map1 could be recovered from map2.
\subsection{learning-based methods}
The last category of background subtraction methods apply traditional machine learning or deep learning on different features for the background modeling.

%SVM
Traditional machine learning methods are commonly involved with support vector machines (SVM) and Bayesian methods. For example, in [], the authors integrate gradient, color, and Haar-like features to address the spatiotemporal variations for each pixel. A pixel wise background model is obtained for each feature in a kernel density framework and a SVM is employed for segmentation.

%
Recent years, deep learning start to flourish in many fields, such as face recognition and natural language processing, significantly improving the state-of-the-art. A novel approach for background subtraction with the use of CNN was proposed by Braham and Droogenbroeck. They employ a scene-specific CNN, which is trained with corresponding image patches from the background image, video frames and groundtruth, or alternatively, segmentation results coming from other background subtraction methods. Patches are extracted around a pixel, then they are feed into the network and compared with a threshold.
%
In [], Yi Wang et al. tried a CNN architecture combined with a Cascade model for segmentation in Background subtraction. Given 200 labelled images as training set, their model performed excellently in dataset2014.
%
In [], M. Babaee et al. present a novel CNN for background subtraction. They also combine the segmentation mask from SuBSENSE algorithm and the output of Flux Tensor algorithm, which is able to adaptively update the parameters used in the background model based on the motion changes in video stream. They also used spatial-median filtering as the post processing of the network outputs.
\section{VARIATION TRANSFORMATION}
In this section, we will elaborate on the motivation of the proposed approach and how does the variation transformation works in background subtraction.

Background subtraction is essentially an binary classification in time sequence, where pixels in a video stream are separated into two categories: foreground (cars, people or animals) and background (roads, trees or other backdrops). 
In most cases, if we print out the historical observations of a single pixel, it is not hard to see that they usually keep their stability when belonging to the background. 
However, there are some exceptions: historical observations varies regularly when belonging to some dynamic background scenes(e.g. 
waves, swaying tree leaves). 
Generally, pixels from the background are sharing some common patterns of variation. 


\begin{figure}[!t]	% FIGURE: figure/fig1 
\centering
    \includegraphics[width=\linewidth]{figure/fig1}
    % 图例里描述一下。
    \caption{It is hard to separate the foreground pixels from the background precisely in a time sequence.}
    \label{variation_chart}
\end{figure}



To give an intuitive understanding of the pixel variations, we plot the historical observations and corresponding groundtruth of a single pixel in Figure.1, with the X axis showing their range of variation and the Y axis representing the time. 
As we can see, there are several lines in three colors: blue, red and green. 
The blue one is called the pixel variation curve, which containing 100 original observations. 
Accordingly, the red one, the groundtruth curve is made of a piece of groundtruth data corresponding to the observations. 
The green one, which we call the transformed variation curve, is made of the outputs of the proposed approach. 
In practice, background subtraction is to produce a prediction curve, which is as near as possible to the groundtruth carve, by given the pixel variation carve. 


Under ideal conditions like indoor videos, previous methods perform quite effectively, when distributions of background and foreground observations are remarkably different and the backgrounds are normally keeping static. 
While in fact, backgrounds can be rather turbulent and dynamic due to the complexity and diversity of nature scenes. 
Especially when illumination and camouflages are involved, background and foreground observations are easily confused and mixed up. 


For example, in Figure.1, it is noticeable that some foreground observations in blue line share the similarly value with background ones. 
In that case, those popular solutions will inevitably yield some sticky moments when separating the pixels in the blue line. 
Statistical methods, for instance, are no longer valid, because they only focus on establishing a statistical model for the background, while having little or no concern for the temporal coherence of these observations. 
Unfortunately, most previous methods, as far as we know, are not capable to take advantage of the temporal coherence of pixel variation. 
In other words, despite knowing that pixels from the background share some common patterns of variation in a temporal sequence, we still let the order information of sequential images all go to waste. 
To alleviate this, order information of pixels must be taken in consideration. 
More concretely, we must find an efficient method to model the patterns of background pixels variations.

In this paper, a method of background subtraction based on variation transformation are proposed. 
Pixel observations are no longer considered independent of each other but regarded as a whole, which we call the pixel variation. 
Consequently, the classification of pixels can be viewed as a transformation of pixel variations, from the observation sequences to the prediction sequences. 
In the specific implementation, we trained a FCN to learn a transformation for the pixel variations by mapping them into a new space where it is close to the groundtruth, just like the green line in Figure.1. 
After thresholding, we can easily get the labels of each observation. 
The benefits from variation transformation are evident and clearly seen. 
It is hard to distinguish a foreground observation when its value are similar to the background ones. 
However, classification on the transformed variation is much convenient and intuitive, due to the advantage of temporal information. 
With the aid of deep learning method, the proposed approach can be effectively implemented.

% 公式表达
And the definition of region searching method $G(x,y)$ is shown as follows:
\begin{equation}
    G(x,y) =  \mathop{\argmin}_{}{ \lVert I_t(m,n) - I_b(x,y) \rVert_{1}  } \quad  m,n \in x,y \pm R,
\end{equation}
where $x$ and $y$ is the location of pixels. And $I_t$ is the current frame, where $t$ is the
time index.

%me
\section{BACKGROUND SUBTRACTION VIA DEEP VARIATION TRANSFORMATION}
In this section, we introduce the proposed approach that consists of a novel FCN network for background subtraction. 
We explain the details of the procedures of pixel matrixes, which is a specific form of pixel variation, and the architecture of our network. 


The complete system is illustrated in Figure 1. 
Firstly, we temporally sample the input and ground truth images to generalize the pixel variations, and reshape them into fixed-size matrixes and fed them into the network. 
After reassembling the matrixes into the complete output frame, it is post-processed, yielding the final segmentation of the respective video frame.
\begin{figure*}[!t]	% FIGURE: figure/fig1 
\centering
\includegraphics[width=\textwidth]{figure/fig2}
\DeclareGraphicsExtensions.
    \caption{Fully convolutional networks can efficiently learn to make dense predictions for per-pixel tasks like semantic segmentation.}
    \label{flow_chart}
\end{figure*}



Given different videos, the number of frames are generally different. 
However, the size of our input is fixed, which means a sampling processing is required to keep the length of pixel variations invariant. 
Therefore, image stacks are sampled form the given videos before the pixel variations are extracted. 
The given video and image stacks can be defined as follows: 
\begin{equation}
	Given \  Video=\{ I_1,I_2,I_3,\dots,I_L\} ,
\end{equation}
\begin{equation}
Stack_{(x)}=\{I_x,I_{(x+p)},I_{(x+2p)},\dots,I_L \},\quad p*l=L,\quad 1\leq   x\leq p , 
\end{equation}
Where $I_t$ represent the frame $t$ of the given video. 
And $p$ is an integer number depended on the video frames length. 
For each video, we can produce multiple image stacks and choose one of them for the training. 
In addition, thanks to the temporal sampling, we can get compact pixel historical observations containing more temporal information than the continuous pixel sequences. 
This works well when it comes to some situation where the moving objects keep stationary for a long time. 


In order to make use of the temporal information of pixel historical observations, we regard the sampled observations from a single pixel as a whole, namely the pixel variation. 
After the temporal sampling, a large number of observation sequences, or pixel variations, are extracted from the chosen stack, which is shown as follows:
\begin{equation}
Sq_{(m)}=\{P_1^m,P_2^m,P_3^m,\dots,P_n^m\},
\end{equation}
Where $P_t^m$ denotes the numerical value of pixel $m$ at frame $t$ in the chosen stack. 
Each of the pixel variation is a piece of temporal information which containing the changing patterns of background pixels. 
And each of them is of fixed length $n$. 
It is notable that the quantity of observation sequences is exactly the resolution of the given video, which means an abundance of training data can be obtained with just one image stack. 


Our intention is to provide an end-to-end transformation of pixel variations, based on the strong learning ability of FCNs. 
However, vectors are not appropriate for the network training and learning. 
Besides, we hope the pixel observations can be interacted with their further compatriots in temporal sequence. 
Thus we reshape the variations into pixel matrixes as the input of our network. 
A sample of pixel matrix $H$ is like this:
\begin{equation}
H=P_{i+j*d}=\begin{bmatrix}
 P_1& P_2  &\dots  &P_d \\ 
\vdots &  &  &\vdots \\ 
 P_{1+(d-1)d}& \dots & \dots & P_D
\end{bmatrix},\quad d^2=D,
\end{equation}
Pixel observations are put into the $d*d$ matrix according to the order of top to bottom and left to right. 
The parameter $i$ and $j$ represent the column and row respectively. 
And the parameter $D$ is the total frame length of a variation. 
To put it from another way, pixel matrix is just a specific form of the pixel variation. 
Although the pixel matrix and the observation sequence belong to different forms of the pixel variation, they are formed of the same components, and share the common temporal information from the pixel variation. 
Since the pixel sequences are extracted from individual pixels, a large number of pixel matrixes may be extracted from only a single image stack.

In the previous steps, videos are broken down into pixel matrixes which containing abundant background information. 
Next, the groundtruth matrixes are obtained in the same way. 
Both of them are put into the network for the training. 
However, the size of the input will decreased after the network computing. 
In order to make output the same size as input, we borrowed ideas from Image semantic segmentation, which is doing padding to the input before the training. 
After the forward computing, variations are transformed in a new space where they can be easily classified by thresholding. 
The new representation of input pixel variation is defined as:
\begin{equation}
H_p= \mathcal L (E_x (H)),
\end{equation}
Where, $H_p$ denotes the output of network, which we call the prediction matrix, the forward computation of network is represented by $L$, and $E_x$ denotes the symmetric padding of pixel matrix where we have lost none of the pixel information to make the output $H_p$ the same size as the pixel matrix $H$. 
Another advantage of symmetric padding is that the order information still remains.

For the loss function, we choose the Sigmoid Cross Entropy (SCE), which are helpful to address the learning slowdown. 
The formulation is as follows:
\begin{equation}
    \begin{aligned}
        & \ell_{H_p, H_{gt}} =  \sum\limits_{x \in X, y \in Y}^{} H_{gt}(x,y) log(Sigmoid(H_p(x,y)))  \\
        & + (1 - H_{gt}(x,y)) log(1 - Sigmoid(H_p(x,y))) \\
        & Sigmoid(x) =\frac{1}{1+e^{-x}}\  ,
    \end{aligned}
\end{equation}

Where $H_{gt}$ denotes the groundtruth matrix which is given by the corresponding GT stack. 
The SCE is calculated between the transformed matrixes and the corresponding groundtruth matrixes. 
Boundaries of moving objects and pixels that out of the region of interest are ignored in the cost function.
Finally, we get the transformed variation through the FCN, which are easier for classification. 
We globally threshold the values for each observation in order to map them to $\{0,1\}$ . 
The threshold function is given by
\begin{equation}
    \label{piecewise_fg}
    g(x,y) =
 \begin{cases}
  1,  \quad x < y       \\
  0,  \quad otherwise   \\
\end{cases}.
\end{equation}
\begin{equation}
M(x,y)=g(H_p,r) ,
\end{equation}
After the thresholding calculations, our experiment results show that a random initialized FCNs, trained end-to-end on feature learning can achieve the state-of-the-art without further machinery. 
And the major contribution is that we demonstrate the effectiveness of temporal information in background 
subtraction.

Here we make a detailed introduction to the deep learning model we used and explain why we choose it.

Different with CNNs, fully convolution neural networks (FCNs) utilize convolutional layers with $1*1$ kernels to take the place of fully connected layers, which largely resolves these above-mentioned problems. 
First and foremost, the size of outputs are adjustable in FCNs, which allows an end-to-end mapping of pixel observation sequences and network outputs on the time sequence. 
We hope to determine the label of a pixel through the comparison of its compatriot in time sequence. 
There is one more point I ought to touch on, that since the feature maps are no longer need to be converted into vectors, spatial information can be retained . 
The last but not the least, FCNs have been used in sematic segmentation and researchers found FCNs have a strong learning ability which won’t lost to the traditional ones. 
Meanwhile, it’s also a high efficient computation model. 


Based on above-mentioned factors, FCN is designed as the alternative network architecture in this paper. 
The structure of our FCN for background modeling is shown in Fig.1. 
The proposed FCN contains 5 convolutional layers, 2 pool layers and a convolutional layer which have a filter size of $1*1$. 
We use the Rectified Linear Unit (ReLU) as activation function after each convolutional layer and the Sigmoid function after the last fully connected layer. 
We do not use any other tricks in our network training and the experiment results prove that the proposed approach is feasible and very effective.


\section{Experiments}

In this section, we ran comprehensive experiments to evaluate the performance of the proposed approach on the CDnet 2014 benchmark and CAMO-UOW. 
The CDnet is the largest dataset for background subtraction so far as we are aware, containing 11 categories with several complexly challenging scenes, such as Dynamic Background, Camera Jitter, Shadow, Night Videos, PTZ and so on. 
The CAMO-UOW is another challenging benchmark which contains 10 high resolution videos. 
For each video, one or two persons appear in the scene with the clothes in the similar color as the background.

The proposed approach is compared with several existing traditional state-of-the-art background subtraction algorithms, including the IUTIS-5, the SuBSENSE, the WeSamBE, sharable GMM, the SharedModel, word-dictionaries-based method and PAWCS, the SemanticBGS, the AAPSA, etc. 
Moreover, two deep learning based algorithms are also compared with the proposed approach, which include DeepBS, and DBMF. 
All the results of compared algorithms are provided by authors.

During the comparison, the F-measure(Fm) has been used for evaluation. 
The Fm is a general international standard in background subtraction which measures the segmentation accuracy by considering both the recall and the precision. 
The definition of Fm is shown as follows:
\begin{equation}
Fm= \frac{2\times precision \times recall}{precision + recall} = \frac{2TP}{2TP+FN+FP},
\end{equation}
% 
% 
% \begin{equation}
%     F=\frac{precision \dot recall}{precision - recall}  = \frac{2TP}{2TP+FN+FP},
% \end{equation}
%
Where TP, FP, and FN are true positives, false positives, and false negatives respectively, computed in pixels of all test frames for each video. 


The quantitative and qualitative comparisons are shown in \reftab{tab1} and Fig.3 respectively. 
Due to the paper length, several typical videos are selected for the qualitative comparisons as well as the discussion. 
In the dynamic background scene, the video “canoe” is a typically challenging video which includes a large area of water rippling. 
The main challenge comes from the dynamic background, in which it is so hard to describe the background by a single image. 
In this condition, since the traditional background subtraction method such as the SharedModel and the WeSamBE do not have the enough ability to describe the complex dynamically background, they are fail to detect the people on the boat, as shown in the Fig. 
X. 
Besides, the detected moving objects of the SharedModel are not accurate in boundary due to the utilization of texture features. 
In contrast, benefited from the strong learning ability of Deep Learning network, the DeepBS successfully detected the people. 
Unfortunately, since the DeepBS ignores the fact that a single background is not enough to describe the dynamic background, even the deep learning based algorithm is suffering from the detection of the boat shape. 
In contrast, the proposed approach performed superior than others in this scene, since the essence of background subtraction is considered as a binary classification of pixels’ observation in time sequence. 
Based on this insight, the FCN network focuses on learning the patterns of the pixel variation rather than a static background image, and proposed approach achieves promising performance in the canoe video.

\begin{figure*}[!t]	\centering
    \includegraphics[width=\textwidth]{figure/fig3}
\DeclareGraphicsExtensions.
    \caption{The qualitative evaluation of the proposed method. All the results is followed in the CDnet 2014.}
    \label{results_chart}
\end{figure*}
As for the case of shadow scene, ‘peopleInShade’ is a typical example with prevalent hard and soft shadows. 
In the traditional approaches, these shadow regions are usually segmented as foreground since it is also moving with the objects. 
Therefore, traditional methods like the PAWCS, the WeSamBE and the SharedModel falsely segments part of shadows as moving objects. 
In addition, the foreground provided by the UBSS is incomplete on the part of the pedestrian’s body due to the interference of shades, which can be owing to the severe dependency of texture features. 
Whereas, the DeepBS performs well in this video benefited from the utilization of CNN. 
However, the shape of pedestrians are slightly deformed as the result of their matrix-wise processing of CNN. 
In contrast, derived from the fact that our DPVL focus on learning the pattern of pixels’ variation in the shadow regions, proposed approach successfully segments the shadow part as the background and achieves the highest performance in the category of shadow scene.

In the video “corridor” among the Thermal scene, there is no color information since the videos are obtained through a Thermal camera. 
Moreover, the moving objects in these videos are exceedingly fuzzy and indistinct, which is the main challenge of this category. 
The WeSamBE, the SharedModel and the PAWCS successfully detect the target objects, owing to a stable background in this indoor video. 
However, they fail to remove the reflections since their modeling ability have already reached a limit under the extreme condition of thermal map. 
The DeepBS, by contrast, succeeds in eliminating most of the reflections. 
Meanwhile, the moving objects are also clearly divided from the background thanks to the strong modeling ability of CNN. 
However, due to the dependency of edge feature, a small object were missed in the detection result. 
Fortunately, the proposed approach focus on the pattern of pixels’ variation, which should be theoretically effective even in the observation without the color information. 
Consequently, our DPVL performed much better than compared algorithms, with the situation that most parts of shadow are removed and the segmentation results are more accurate.

% \begin{table}[tab1]
\begin{table*}[!t]				% TABLE
\centering
\caption{The performance comparison of the proposed approach and some state-of-the-art algorithms on the video sequences from different categories in CDnet 2014.}
\label{tab1}
\begin{tabular}{lllllllllllll}
\hline
Videos      & baseline & dyna.bg & cam.jitter & int.obj.m & shadow & thermal & bad.weat & low f.rate & night vid. & PTZ    & turbul. & overall \\ \hline
DeepBS      & 0.9580   & 0.8761  & \textbf{0.8990}     & 0.6097    & 0.9304 & 0.7583  & 0.8647   & 0.5900     & 0.6359     & 0.3306 & \textbf{0.8993}  & 0.7458  \\
IUTIS-5     & 0.9567   & 0.8902  & 0.8332     & 0.7296    & 0.9084 & 0.8303  & 0.8289   & \textbf{0.7911}     & 0.5132     & 0.4703 & 0.8507  & 0.7717  \\
FTSG        & 0.9330   & 0.8792  & 0.7513     & 0.7891    & 0.8832 & 0.7768  & 0.8228   & 0.6259     & 0.5130     & 0.3241 & 0.7127  & 0.7283  \\
AAPSA       & 0.9183   & 0.6706  & 0.7207     & 0.5098    & 0.7953 & 0.7030  & 0.7742   & 0.4942     & 0.4161     & 0.3302 & 0.4643  & 0.6179  \\
CwisarDH    & 0.9145   & 0.8274  & 0.7886     & 0.5753    & 0.8581 & 0.7866  & 0.6837   & 0.6406     & 0.3735     & 0.3218 & 0.7227  & 0.6812  \\
PAWCS       & 0.9397   & 0.8938  & 0.8137     & 0.7764    & 0.8934 & 0.8324  & 0.8059   & 0.6433     & 0.4171     & 0.4450 & 0.7667  & 0.7403  \\
SuBSENSE    & 0.9503   & 0.8177  & 0.8152     & 0.6569    & 0.8986 & 0.8171  & 0.8594   & 0.6594     & 0.4918     & 0.3894 & 0.8423  & 0.7408  \\
SemanticBGS & 0.9604   & \textbf{0.9489}  & 0.8388     & 0.7878    & 0.9244 & 0.8219  & 0.8260   & 0.7888     & 0.5014     & 0.5673 & 0.6921  & 0.7892  \\
MBS         & 0.9287   & 0.7915  & 0.8367     & 0.7568    & 0.8262 & 0.8194  & 0.7980   & 0.6350     & 0.5158     & 0.5520 & 0.5858  & 0.7288  \\
WeSamBE     & 0.9413   & 0.7440  & 0.7976     & 0.7392    & 0.8999 & 0.7962  & 0.8608   & 0.6602     & 0.5929     & 0.3844 & 0.7737  & 0.7446  \\
ShareM      & 0.9522   & 0.8222  & 0.8141     & 0.6727    & 0.8898 & 0.8319  & 0.8480   & 0.7286     & 0.5419     & 0.3860 & 0.7339  & 0.7474  \\
GMM         & 0.8245   & 0.633   & 0.5969     & 0.5207    & 0.7370 & 0.6621  & 0.7380   & 0.5373     & 0.4097     & 0.1522 & 0.4663  & 0.5707  \\
RMoG        & 0.7848   & 0.7352  & 0.7010     & 0.5431    & 0.7212 & 0.4788  & 0.6826   & 0.5312     & 0.4265     & 0.2470 & 0.4578  & 0.5735  \\ \hline
Ours        & \textbf{0.9668}   & 0.8639  & 0.8446     & \textbf{0.9181}    & \textbf{0.9390} & \textbf{0.9268}  & \textbf{0.9265}   & 0.7338     & \textbf{0.7306}     & \textbf{0.5845} & 0.8761  & \textbf{0.8504} \\ \hline
\end{tabular}
\end{table*}

\begin{table*}[!t]
\centering
\caption{The performance comparison of the proposed approach and some classical methods and deep-based method DBMF .}
\label{tab2}
\begin{tabular}{llllllll}
\hline
Methods  & highway & office & Pedestrians & PETS2006 & Fall   & sofa   & overall \\ \hline
GMM      & 0.5788  & 0.2338 & 0.5202      & 0.6011   & 0.8026 & 0.5225 & 0.5432  \\
CodeBook & 0.8356  & 0.5939 & 0.7293      & 0.7808   & 0.3921 & 0.8149 & 0.6911  \\
ViBe     & 0.7535  & 0.6676 & 0.8367      & 0.6668   & 0.6829 & 0.4298 & 0.6729  \\
PBAS     & 0.8071  & 0.6839 & 0.7902      & 0.7280   & 0.3420 & 0.5768 & 0.6547  \\
P2M      & 0.9160  & 0.3849 & 0.9121      & 0.7322   & 0.5819 & 0.4352 & 0.6604  \\
DBMF     & 0.9412  & 0.9236 & 0.8394      & 0.9059   & 0.8203 & 0.8645 & 0.8824  \\ \hline
ours     & \textbf{0.9775}  & \textbf{0.9565} & \textbf{0.9652}      & \textbf{0.9681}   & \textbf{0.9157} & \textbf{0.8943} & \textbf{0.9462}  \\ \hline
\end{tabular}
\end{table*}

    
    \begin{table*}[!t]
\centering
\caption{The performance comparison of the proposed approach and some state-of-the-art algorithms on the video sequences from different categories in CAMO-UOW.}
\label{tab3}
\begin{tabular}{lllllllllll}
\hline
Methods  & MOG2 & FCI  & LBA-SOM & PBAS & SuBSENSE & ML-BGS & DECOLOR       & COROLA & FWFC          & Ours          \\ \hline
Video 1  & 0.79 & 0.88 & 0.8     & 0.9  & 0.89     & 0.89   & 0.92          & 0.8    & \textbf{0.94} & \textbf{0.94} \\
Video 2  & 0.82 & 0.79 & 0.8     & 0.82 & 0.88     & 0.8    & 0.83          & 0.58   & 0.96          & \textbf{0.98} \\
Video 3  & 0.88 & 0.86 & 0.85    & 0.91 & 0.9      & 0.8    & 0.9           & 0.82   & \textbf{0.94} & \textbf{0.94} \\
Video 4  & 0.89 & 0.9  & 0.76    & 0.93 & 0.78     & 0.88   & 0.95          & 0.87   & 0.94          & \textbf{0.97} \\
Video 5  & 0.84 & 0.86 & 0.82    & 0.83 & 0.82     & 0.8    & 0.82          & 0.75   & 0.91          & \textbf{0.97} \\
Video 6  & 0.93 & 0.87 & 0.77    & 0.95 & 0.92     & 0.95   & \textbf{0.97} & 0.72   & 0.94          & 0.96          \\
Video 7  & 0.76 & 0.83 & 0.88    & 0.91 & 0.87     & 0.79   & 0.91          & 0.83   & 0.96          & \textbf{0.99} \\
Video 8  & 0.83 & 0.87 & 0.85    & 0.87 & 0.93     & 0.86   & 0.86          & 0.68   & 0.96          & \textbf{0.98} \\
Video 9  & 0.89 & 0.9  & 0.87    & 0.84 & 0.92     & 0.87   & 0.86          & 0.78   & 0.88          & \textbf{0.99} \\
Video 10 & 0.89 & 0.86 & 0.89    & 0.91 & 0.92     & 0.9    & 0.94          & 0.85   & 0.96          & \textbf{0.97} \\ \hline
average  & 0.85 & 0.86 & 0.83    & 0.89 & 0.88     & 0.85   & 0.90          & 0.77   & 0.94          & \textbf{0.97} \\ \hline
\end{tabular}
\end{table*}




The quantitative evaluation of proposed approach on CDnet 2014 is shown in the \reftab{tab1}. 
It can be inferred that the proposed approach significantly outperformed all of the compared state-of-the-art algorithms in most of complex scenes and achieved 6\% gain in FM over the second one on the whole dataset. 
Moreover, in order to compare proposed approach with the DBMF, which is also based on deep learning and only publish their results in several special vides. 
The proposed approach has also ran in these video and the results are shown in Table 2. 
Again, the proposed approach has noticeably better performance than the DBMF and some other classical background approaches.

	As shown in the \reftab{tab1} and \reftab{tab2}, previous deep learning based methods like the DeepBS and the DBMF achieve well performance. 
From our own perspective, that good performance should attribute to the stronger modeling ability and learning adaptation of CNN and FCN. 
However, the proposed approach focused on the pixels variation in temporal sequence rather than low-level static features such as color, edges and textures, which gives us the ability to avoid the shortcomings of the background models. 
Consequently, the proposed approach still get considerably better results, which over 10.46\% in FM metrics compared with the DeepBS and over 6.38\% compared with the DBMF.
The evaluation of proposed approach in CAMO-UOW dataset is shown in the \reftab{tab3}. 
Unlike the CDnet dataset, the videos of CAMO-UOW dataset are specially proposed for the moving objects with camouflage, which is the main challenge of this dataset. 
As shown in the \reftab{tab3} proposed approach achieves better performance compared to its competitions, with an average F-measure of 0.97, compared to values between 0.77 and 0.94 for the other methods. 
Therefore, it is fair to say that proposed approach performs better compared with their peers. 


In this dataset, target objects have the similar color and textures with the background, which brings a lot of difficulties and obstacles to traditional methods. 
However, our FCN is a powerful Neural Network model which is good at capturing the non-linearities of the manifold of pixel variations. 


All these experiments of the proposed method were implemented in matlab and ran on the computer with Nvidia tasela K80 GPU and all images are keep their original resolution. 
For each video in CDnet 2014, 100 training frames are extracted to produce the image matrix. 
It should be noted that the 100 frames only accounts for less than 10\% of total Groundtruth in CDnet 2014. 
In contrast, 90\% of data were used as training samples in the DBMF, which suggest that the proposed approach achieves well performance with limited training frames. 
Considering that the videos in CAMO-UOW have fewer frames, we reduce the number of training frames to 49. 
During the experience, the training set and testing set are completely separated. 
More specifically, our FCN network is random initialized. 
We train the network with mini-batches of size 100, a learning rate $α = 1 ∗ 10^{−3}$ over 20 epochs. 
The last threshold $R$ is set to 0.6.


% \begin{table}[!t]				% TABLE
%     \caption{The example of latex table}
% \label{tab_FBMS_nobk}
% \centering
%     \begin{tabular}{|l@{  }c@{  }c@{  }cc@{  }c@{  }c@{  }c|}
% \hline
%      \multirow{2}{*}{Videos} & \multicolumn{3}{c}{$\text{IFB}_{nobk}$} &  & \multicolumn{3}{c|}{IFB} \\
%     \cline{2-4} \cline{6-8}
%        & Re &Pr & Fm &  &  Re & Pr & Fm  \\
% \hline
% cats03          &   (\textbf{0.9472} ,\   &  0.5428 ,\  &   0.6901 )   &  &    (0.9063 ,\	 &  \textbf{0.7955} ,\	 &  \textbf{0.8473} )   \\
% bear01          &   (\textbf{0.9854} ,\   &  0.4806 ,\  &   0.6461 )   &  &    (0.8695 ,\	 &  \textbf{0.8673} ,\	 &  \textbf{0.8684} )   \\
% rabbits01       &   (\textbf{0.9530} ,\   &  0.4433 ,\  &   0.6051 )   &  &    (0.9510 ,\	 &  \textbf{0.8730} ,\	 &  \textbf{0.9103} )   \\
% cars5           &   (\textbf{0.9887} ,\   &  0.3693 ,\  &   0.5378 )   &  &    (0.9437 ,\	 &  \textbf{0.7108} ,\	 &  \textbf{0.8109} )   \\
% tennis          &   (\textbf{0.9364} ,\   &  0.4558 ,\  &   0.6132 )   &  &    (0.8808 ,\	 &  \textbf{0.7345} ,\	 &  \textbf{0.8010} )   \\
% farm01          &   (\textbf{0.9887} ,\   &  0.3897 ,\  &   0.5591 )   &  &    (0.8954 ,\	 &  \textbf{0.7462} ,\	 &  \textbf{0.8140} )   \\
% \hline                                                                                                           
% Average         &   (\textbf{0.9666} ,\   &  0.4469 ,\  &   0.6086 )   &  &    (0.9078 ,\    &  \textbf{0.7879} ,\   &  \textbf{0.8420} )   \\
% \hline
% \end{tabular}
% \end{table}

% Videos & GBSSP\cite{Lim2014_2014_ECCV} & calMoSeg\cite{2016_ECCV_Bideau2016}  & MLayer\cite{2017_ICCV_zhu2017multilayer}  & $\text{IFB}_{SU}$ & $\text{IFB}_{KA}$ & $\text{IFB}_{SI}$ & Videos & GBSSP\cite{Lim2014_2014_ECCV} & calMoSeg\cite{2016_ECCV_Bideau2016} & MLayer\cite{2017_ICCV_zhu2017multilayer}  &  $\text{IFB}_{SU}$ &  $\text{IFB}_{KA}$  &  $\text{IFB}_{SI}$  \\




\section{Conclusion}

% \section{Conclusions}
% In this paper,
% we proposed the IFB framework for background subtraction for the case of a freely moving camera.
% Unlike previous work in which attempts were made to improve the accuracy of the estimation of motion,
% our IFB focuses on integrating rough foreground and background cues for foreground segmentation.
% In particular, 
% foreground cues are detected by a GMM model with the estimation of background motion,
% while background cues are captured from spatio-temporal features filtered by homography transformation,
% where the SURF\cite{2006_SURF}, KAZE\cite{2012_KAZE}, SIFT\cite{lowe2004distinctive} features are used as examples.
% % the geometric constraints between the SIFT features.
% Then, super-pixels under multiple levels are utilized to integrate these cues.
% The efficiency of IFB results from the complementarity between foreground and background cues.
% The accuracy of the proposed approach is improved though the utilization of super-pixels under multiple levels.
% A comprehensive experiment to compare our results with the state-of-the-art shows the efficiency of our framework and points to its potential for use in practical applications.
% 
\ifCLASSOPTIONcaptionsoff
  \newpage
\fi

\bibliographystyle{IEEEtran}  
\bibliography{ref}  


\end{document}
